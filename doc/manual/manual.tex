
% This is LLNCS.DEM the demonstration file of
% the LaTeX macro package from Springer-Verlag
% for Lecture Notes in Computer Science,
% version 2.4 for LaTeX2e as of 16. April 2010
%
\documentclass{llncs}
%
%\synctex=1

\usepackage[linesnumbered,ruled,vlined]{algorithm2e}

%\usepackage[T1]{fontenc}
%\usepackage{lmodern}

\usepackage{graphicx}
\usepackage{epsfig}

\usepackage{array}
%\setlength{\extrarowheight}{0pt}


%\usepackage{bigstrut}
%    \setlength\bigstrutjot{3pt}


\usepackage{amssymb}
%\usepackage{MnSymbol}
\usepackage{bm}
\usepackage{amsmath}

\usepackage{bussproofs}
\usepackage[USenglish]{babel}

\usepackage{subfigure}

\usepackage{times}

\usepackage{booktabs}

\usepackage{float}

\usepackage{multirow}


\usepackage{todonotes}

%\usepackage[T1]{fontenc}
%\usepackage{cmbright}


% \fontencoding{T1}
%  \fontfamily{garamond}
%  \fontseries{m}
%  \fontshape{it}
%  \fontsize{12}{15}
%  \selectfont

% correct bad hyphenation here
\hyphenation{op-tical net-works semi-conduc-tor}

\newcommand{\code}[1]{{\fontfamily{cmtt}\small\selectfont#1}}
\newcommand{\codefs}[1]{{\fontfamily{cmtt}\scriptsize\selectfont#1}}


%\usepackage{fancyvrb}
%\DefineVerbatimEnvironment{codenv}{Verbatim}{fontfamily=ssf}
%cmss
%{\begin{Verbatim}[fontfamily=cmss]}% 
%{\end{Verbatim}}%

% multiline comments
\usepackage{verbatim} 

\usepackage{color}
\usepackage{listings}

%  basicstyle=\sffamily\small,


\lstdefinelanguage{Scala}{
	morekeywords={},
	morekeywords=[2]{
		public, private, protected,
		abstract,case,catch,class,def,
		do,else,extends,false,final,finally,
		for,if,implicit,import,match,mixin,
		new,null,object,override,package,
		private,protected,requires,return,sealed,
		super,this,throw,trait,true,try,
		type,val,var,while,with,yield,
		lazy,evt,observable,imperative,
                after,before},
	otherkeywords={=>,<-,<\%,<:,>:,\#,@},
	sensitive=true,
	morecomment=[l]{//},
	morecomment=[n]{/*}{*/},
	morestring=[b]",
	morestring=[b]',
	morestring=[b]"""
}
\lstloadlanguages{Scala}

\lstset{
  backgroundcolor=\color{white},%\fontfamily{cmtt}
  basicstyle=\fontfamily{cmtt}\scriptsize,
  basewidth=0.5em,
  showstringspaces=false,
  keywordstyle=\fontfamily{pcr}\color[rgb]{0,0,0}\bfseries,
  %commentstyle=\color[rgb]{0.133,0.545,0.133},
  %stringstyle=\color[rgb]{0.627,0.126,0.941},
  breaklines=false,
  frame=none
  breaklines=false,
  frame=none,
  numbers=left, numberstyle=\tiny, stepnumber=1, numbersep=5pt,
  %frameround=fttt,
  %frame=single,
  escapeinside={(*@}{@*)},
  %columns=fullflexible
}


\lstnewenvironment{codenv}{\lstset{language=Scala}}{}
\lstnewenvironment{codenvOpt}[1]{\lstset{language=Scala,#1}}{}


\newenvironment{bluetext}{\color{blue}}{\ignorespacesafterend}
\newenvironment{redtext}{\color{red}}{\ignorespacesafterend}
\newcommand{\bygerold}[1]{ \begin{bluetext} #1 \end{bluetext}}
\newcommand{\commentbygerold}[1]{\\\begin{redtext} #1 \end{redtext}\\}

%\newenvironment{codenv}{\small \begin{itshape}}{ \end{itshape}}



%\renewcommand*{\bibfont}{\footnotesize}



% Avoid (not that) big pictures alone in a blank page
\renewcommand{\topfraction}{0.85}
\renewcommand{\textfraction}{0.1}
\renewcommand{\floatpagefraction}{0.75}

\usepackage{xspace}
\newcommand{\REScala}{{\small \sc{REScala}}\xspace}


\newcommand{\mm}[1]{\todo[color=green!40]{#1}}
\newcommand{\jd}[1]{\todo[color=blue!40]{#1}}
\newcommand{\gs}[1]{\todo[color=red!40]{#1}}

\usepackage{color}


\hyphenation{Res-ca-la}


\newcommand{\str}[1]{{\sf\selectfont @#1@}\newline}


\begin{document}

% Allows default copyright year (200X to be over-ridden - IF NEED BE.

% Allows default copyright data (0-89791-88-6/97/05) to be over-ridden.
% --- End of Author Metadata ---

%\copyrightdata{XXXXXX}


\pagestyle{headings}  % switches on printing of running heads
%\pagestyle{plain}  
%
\title{REScala Users Manual}
%
%\titlerunning{}
% abbreviated title (for running head)
%                                     also used for the TOC unless
%                                     \toctitle is used
%
\author{Guido Salvaneschi\\
with\\
Gerold Hintz, Pascal Weisenburger}
%
\authorrunning{Salvaneschi et
  al.} % abbreviated author list (for running head)
%

\institute{
%  Software Technology Group\\
  Technische Universit\"at Darmstadt\\
\code{salvaneschi@informatik.tu-darmstadt.de}
% \and Software Technology Group\\
% Technische Universit\"at Darmstadt
% \code{mezini@informatik.tu-darmstadt.de}
}

\maketitle              % typeset the title of the contribution







\begin{abstract}
Intro
\end{abstract}

%\category{D.1.3}{Software}{Programming Techniques}[Concurrent
%Programming]\category{\hspace{-2mm}D.3.3}{Programming
%  Languages}{Language Constructs and Features}

%\terms{Languages, Design}

\keywords Functional-reactive Programming, Scala, Event-driven
Programming

\section{Signals and Vars}



\section{Events}

\section{Imperative events}


\section{Declarative Events}


\section{Conversion Functions}




\section{Related Work}~\label{sec:related}

REScala builds on ideas originally developed in
EScala~\cite{Gasiunas:2011:EME:1960275.1960303} -- which supports
event combination and implicit events.

Other reactive languages directly represent time-changing values and
remove inversion of control. Among the others, we mention
FrTime~\cite{DBLP:conf/esop/CooperK06} (Scheme),
FlapJax~\cite{Meyerovich:2009:FPL:1640089.1640091} (Javascript),
AmbientTalk/R~\cite{ambienttalkR} and
Scala.React~\cite{EPFL-REPORT-148043} (Scala).


\section{Acknowledgments}
This work has been supported by the German Federal Ministry of
Education and Research (Bundesministerium f\"ur Bildung und Forschung,
BMBF) under grant No.\\
16BY1206E and by the European Research Council, grant No. 321217.

\bibliographystyle{abbrv}
\bibliography{report}




\end{document}














